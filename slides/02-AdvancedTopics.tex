\documentclass[ignorenonframetext,]{beamer}
\usetheme{Darmstadt}
\usecolortheme{beaver}
\usefonttheme{structurebold}
\setbeamertemplate{caption}[numbered]
\setbeamertemplate{caption label separator}{:}
\setbeamercolor{caption name}{fg=normal text.fg}
\usepackage{amssymb,amsmath}
\usepackage{ifxetex,ifluatex}
\usepackage{fixltx2e} % provides \textsubscript
\usepackage{lmodern}
\ifxetex
  \usepackage{fontspec,xltxtra,xunicode}
  \defaultfontfeatures{Mapping=tex-text,Scale=MatchLowercase}
  \newcommand{\euro}{€}
\else
  \ifluatex
    \usepackage{fontspec}
    \defaultfontfeatures{Mapping=tex-text,Scale=MatchLowercase}
    \newcommand{\euro}{€}
  \else
    \usepackage[T1]{fontenc}
    \usepackage[utf8]{inputenc}
      \fi
\fi
% use upquote if available, for straight quotes in verbatim environments
\IfFileExists{upquote.sty}{\usepackage{upquote}}{}
% use microtype if available
\IfFileExists{microtype.sty}{\usepackage{microtype}}{}
\usepackage{color}
\usepackage{fancyvrb}
\newcommand{\VerbBar}{|}
\newcommand{\VERB}{\Verb[commandchars=\\\{\}]}
\DefineVerbatimEnvironment{Highlighting}{Verbatim}{commandchars=\\\{\}}
% Add ',fontsize=\small' for more characters per line
\usepackage{framed}
\definecolor{shadecolor}{RGB}{248,248,248}
\newenvironment{Shaded}{\begin{snugshade}}{\end{snugshade}}
\newcommand{\KeywordTok}[1]{\textcolor[rgb]{0.13,0.29,0.53}{\textbf{{#1}}}}
\newcommand{\DataTypeTok}[1]{\textcolor[rgb]{0.13,0.29,0.53}{{#1}}}
\newcommand{\DecValTok}[1]{\textcolor[rgb]{0.00,0.00,0.81}{{#1}}}
\newcommand{\BaseNTok}[1]{\textcolor[rgb]{0.00,0.00,0.81}{{#1}}}
\newcommand{\FloatTok}[1]{\textcolor[rgb]{0.00,0.00,0.81}{{#1}}}
\newcommand{\CharTok}[1]{\textcolor[rgb]{0.31,0.60,0.02}{{#1}}}
\newcommand{\StringTok}[1]{\textcolor[rgb]{0.31,0.60,0.02}{{#1}}}
\newcommand{\CommentTok}[1]{\textcolor[rgb]{0.56,0.35,0.01}{\textit{{#1}}}}
\newcommand{\OtherTok}[1]{\textcolor[rgb]{0.56,0.35,0.01}{{#1}}}
\newcommand{\AlertTok}[1]{\textcolor[rgb]{0.94,0.16,0.16}{{#1}}}
\newcommand{\FunctionTok}[1]{\textcolor[rgb]{0.00,0.00,0.00}{{#1}}}
\newcommand{\RegionMarkerTok}[1]{{#1}}
\newcommand{\ErrorTok}[1]{\textbf{{#1}}}
\newcommand{\NormalTok}[1]{{#1}}

% Comment these out if you don't want a slide with just the
% part/section/subsection/subsubsection title:
\AtBeginPart{
  \let\insertpartnumber\relax
  \let\partname\relax
  \frame{\partpage}
}
\AtBeginSection{
  \let\insertsectionnumber\relax
  \let\sectionname\relax
  \frame{\sectionpage}
}
\AtBeginSubsection{
  \let\insertsubsectionnumber\relax
  \let\subsectionname\relax
  \frame{\subsectionpage}
}

\setlength{\parindent}{0pt}
\setlength{\parskip}{6pt plus 2pt minus 1pt}
\setlength{\emergencystretch}{3em}  % prevent overfull lines
\setcounter{secnumdepth}{0}
\usepackage{pgfplots}

\title{Advanced Research Tools for Economics and Business Administration (Part
II)}
\author{Thomas de Graaff}
\date{January 22, 2015}

\begin{document}
\frame{\titlepage}

\section{Where are we}\label{where-are-we}

\begin{frame}{Previous tutorial}

Still somewhat more theoretical (why do you want to change tools)

\begin{itemize}
\item
  Importance of writing things down (reproducability)
\item
  Text files are the bomb:

  \begin{itemize}
  \itemsep1pt\parskip0pt\parsep0pt
  \item
    scriptable
  \item
    input and output in/for other applications
  \end{itemize}
\item
  pros and cons of \LaTeX
\item
  Why bother with learning \LaTeX?

  \begin{itemize}
  \itemsep1pt\parskip0pt\parsep0pt
  \item
    for dead threes (aka paper)
  \item
    html (cloud) uses \LaTeX~syntax as well for formula's and graph
    annotation
  \end{itemize}
\end{itemize}

\end{frame}

\begin{frame}[fragile]{A quick recap}

\begin{itemize}
\item
  Specific \LaTeX~commands starts with an \textbackslash{}

  \begin{itemize}
  \itemsep1pt\parskip0pt\parsep0pt
  \item
    \texttt{\textbackslash{}LaTeX}
  \end{itemize}
\item
  Inline equations are within \$ \$

  \begin{itemize}
  \itemsep1pt\parskip0pt\parsep0pt
  \item
    \texttt{\$\textbackslash{}frac\{a\}\{b\}\$ is the fraction between \$a\$ and \$b\$}
  \end{itemize}
\item
  There are a number of symbols that you cannot immediately use:

  \begin{itemize}
  \itemsep1pt\parskip0pt\parsep0pt
  \item
    \textbackslash{}, \$, \&, \%, \{ and \} are the most important
    (solution: start with an \texttt{\textbackslash{}})
  \end{itemize}
\item
  Environments start and end

\begin{Shaded}
\begin{Highlighting}[]
\NormalTok{\textbackslash{}begin\{equation\} }
\NormalTok{a^2 + b^2 = c^2 }
\NormalTok{\textbackslash{}end\{equation\}}
\end{Highlighting}
\end{Shaded}
\end{itemize}

\end{frame}

\begin{frame}[fragile]{General structure}

\begin{Shaded}
\begin{Highlighting}[]
\NormalTok{\textbackslash{}documentclass[twocolumn, a4paper]\{article\}}

\CommentTok{% Preamble: how should it look like}
\NormalTok{\textbackslash{}usepackage\{multicol, lipsum\}}
\NormalTok{\textbackslash{}usepackage[english, german]\{babel\}}

\NormalTok{\textbackslash{}begin\{document\}}
    \CommentTok{% Body: the real contents}
    \NormalTok{\textbackslash{}lipsum}
\NormalTok{\textbackslash{}end\{document\}}
\end{Highlighting}
\end{Shaded}

\end{frame}

\begin{frame}{This tutorial}

More practical, play around with \LaTeX. In specific:

\begin{itemize}
\itemsep1pt\parskip0pt\parsep0pt
\item
  packages (make things look better)
\item
  figures (usually import them, but sometimeS make them yourself)
\item
  tables (import them!)
\item
  slides (just copy \& paste from \texttt{.tex} document)
\end{itemize}

\end{frame}

\section{Making appearances}\label{making-appearances}

\begin{frame}[fragile]{The use of packages}

\begin{itemize}
\item
  Typically, packages are used to change appearance
\item
  To ensure no errors, usually opt for the full installation or have
  access to internet
\item
  There are lots of them, see \href{http://www.ctan.org}{CTAN}
\item
  Often used packages

  \begin{itemize}
  \itemsep1pt\parskip0pt\parsep0pt
  \item
    amsmath, graphicx, subfig, marvosym, microtype, booktabs, lipsum,
    pdflscape, fullpage
  \end{itemize}
\item
  format:

\begin{Shaded}
\begin{Highlighting}[]
\NormalTok{\textbackslash{}usepackage[colorlinks=true,citecolor=magenta,}
        \NormalTok{urlcolor=magenta]\{hyperref\} }
\end{Highlighting}
\end{Shaded}
\end{itemize}

\end{frame}

\begin{frame}[fragile]{The use of classes}

\begin{itemize}
\item
  Typically one uses the \texttt{article} class
\item
  However, there is as well a \texttt{book}, \texttt{mininal},
  \texttt{report}, \texttt{beamer} class etcetera
\item
  Specific user written classes are \texttt{memoir} and
  \texttt{elsarticle} classes
\item
  Classes come with options such as

\begin{Shaded}
\begin{Highlighting}[]
\NormalTok{\textbackslash{}documentclass[12pt, a4paper]\{article\}}
\end{Highlighting}
\end{Shaded}
\end{itemize}

\end{frame}

\begin{frame}[fragile]{Bibliopgraphy}

Default format is \texttt{BibTeX} - customizable (however limited) -
defaults is good

If you want to customize quite a lot: \texttt{biblatex-biber}
combination - usage

\begin{Shaded}
\begin{Highlighting}[]
    \NormalTok{\textbackslash{}usepackage[backend=biber]\{biblatex\}}
\end{Highlighting}
\end{Shaded}

\end{frame}

\begin{frame}[fragile]{or go nuts}

\begin{Shaded}
\begin{Highlighting}[]
\NormalTok{\textbackslash{}usepackage[style= authoryear-icomp, }
            \NormalTok{backend=bibtex,}
            \NormalTok{natbib=true, }
            \NormalTok{firstinits=true, }
            \NormalTok{uniquename=true,}
            \NormalTok{backref=false,}
            \NormalTok{doi=false,}
            \NormalTok{isbn=false,}
            \NormalTok{url=false,}
            \NormalTok{maxnames=2, }
            \NormalTok{maxbibnames=10, }
            \NormalTok{dashed =true, }
            \NormalTok{backend=biber]}
\NormalTok{            \{biblatex\}}
\end{Highlighting}
\end{Shaded}

\end{frame}

\section{Better graphs}\label{better-graphs}

\begin{frame}[fragile]{Import them}

Remember figures/graphs and tables in a floating environment

\begin{Shaded}
\begin{Highlighting}[]
\NormalTok{\textbackslash{}begin\{figure\}[h!] }
    \NormalTok{\textbackslash{}center }
        \NormalTok{\textbackslash{}includegraphics\{ligatures_latex\} }
    \NormalTok{\textbackslash{}caption\{...\} }
    \NormalTok{\textbackslash{}label\{ligatures\}}
\NormalTok{\textbackslash{}end\{figure\}}
\end{Highlighting}
\end{Shaded}

\begin{itemize}
\itemsep1pt\parskip0pt\parsep0pt
\item
  \texttt{\textbackslash{}ref\{ligatures\}} gives you now the correct
  internal reference
\item
  How to make pictures then:

  \begin{itemize}
  \itemsep1pt\parskip0pt\parsep0pt
  \item
    In the statistical environment you are working in
  \item
    plotly
  \end{itemize}
\end{itemize}

\end{frame}

\begin{frame}[fragile]{Making them yourself in \LaTeX~(advanced)}

PGF/TikZ combination for producing vector graphics

\begin{Shaded}
\begin{Highlighting}[]
\NormalTok{\textbackslash{}begin\{tikzpicture\}}
    \NormalTok{\textbackslash{}begin\{axis\}[}
        \NormalTok{xlabel=$x$,}
        \NormalTok{ylabel=\{$f(x) = x^2 - x +4$\}}
    \NormalTok{]}
    \CommentTok{% use TeX as calculator:}
    \NormalTok{\textbackslash{}addplot \{x^2 - x +4\};}
    \NormalTok{\textbackslash{}end\{axis\}}
\NormalTok{\textbackslash{}end\{tikzpicture\}}
\end{Highlighting}
\end{Shaded}

\end{frame}

\begin{frame}{Which results in}

\begin{tikzpicture}
    \begin{axis}[
        xlabel=$x$,
        ylabel={$f(x) = x^2 - x +4$}
    ]
    % use TeX as calculator:
    \addplot {x^2 - x +4};
    \end{axis}
\end{tikzpicture}

\end{frame}

\section{Better tables}\label{better-tables}

\begin{frame}[fragile]{Some guidelines}

\begin{itemize}
\item
  No vertical lines!
\item
  small spaces are usually better than horizontal lines
\item
  Booktabs is a nice package

\begin{Shaded}
\begin{Highlighting}[]
\NormalTok{\textbackslash{}toprule}
\NormalTok{\textbackslash{}midrule}
\NormalTok{\textbackslash{}addlinespace}
\NormalTok{\textbackslash{}bottomrule}
\end{Highlighting}
\end{Shaded}
\item
  Only include stuff that is important
\end{itemize}

\end{frame}

\begin{frame}[fragile]{This does not look nice!}

\tiny

\begin{Shaded}
\begin{Highlighting}[]
\NormalTok{\textbackslash{}begin\{table\}[htbp]\textbackslash{}centering}
\NormalTok{\textbackslash{}def\textbackslash{}sym#1\{\textbackslash{}ifmmode^\{#1\}\textbackslash{}else\textbackslash{}(^\{#1\}\textbackslash{})\textbackslash{}fi\}}
\NormalTok{\textbackslash{}caption\{Dep = Milles per Gallon\}}
\NormalTok{\textbackslash{}begin\{tabular\}\{l*\{2\}\{D\{.\}\{.\}\{-1\}\}\}}
\NormalTok{\textbackslash{}toprule}
                    \NormalTok{&\textbackslash{}multicolumn\{1\}\{c\}\{(1)\}&\textbackslash{}multicolumn\{1\}\{c\}\{(2)\}\textbackslash{}\textbackslash{}}
                    \NormalTok{&\textbackslash{}multicolumn\{1\}\{c\}\{Mileage (mpg)\}&\textbackslash{}multicolumn\{1\}\{c\}\{Mileage (mpg)\}\textbackslash{}\textbackslash{}}
\NormalTok{\textbackslash{}midrule}
\NormalTok{Car type            &     -1.6500  &     -2.2035* \textbackslash{}\textbackslash{}}
                    \NormalTok{&    (1.0760)  &    (1.0592)  \textbackslash{}\textbackslash{}}
\NormalTok{Weight (lbs.)       &     -0.0066**&     -0.0166**\textbackslash{}\textbackslash{}}
                    \NormalTok{&    (0.0006)  &    (0.0040)  \textbackslash{}\textbackslash{}}
\NormalTok{weight\textbackslash{}_sqr          &              &      0.0000* \textbackslash{}\textbackslash{}}
                    \NormalTok{&              &    (0.0000)  \textbackslash{}\textbackslash{}}
\NormalTok{Constant            &     41.6797**&     56.5388**\textbackslash{}\textbackslash{}}
                    \NormalTok{&    (2.1655)  &    (6.1974)  \textbackslash{}\textbackslash{}}
\NormalTok{\textbackslash{}midrule}
\NormalTok{Observations        &          74  &          74  \textbackslash{}\textbackslash{}}
\NormalTok{\textbackslash{}(R^\{2\}\textbackslash{})           &       0.663  &       0.691  \textbackslash{}\textbackslash{}}
\NormalTok{F                   &     69.7485  &     52.2515  \textbackslash{}\textbackslash{}}
\NormalTok{\textbackslash{}bottomrule}
\NormalTok{\textbackslash{}multicolumn\{3\}\{l\}\{\textbackslash{}footnotesize Standard errors in parentheses\}\textbackslash{}\textbackslash{}}
\NormalTok{\textbackslash{}multicolumn\{3\}\{l\}\{\textbackslash{}footnotesize + \textbackslash{}(p<0.1\textbackslash{}), * \textbackslash{}(p<0.05\textbackslash{}), ** \textbackslash{}(p<0.01\textbackslash{})\}\textbackslash{}\textbackslash{}}
\NormalTok{\textbackslash{}end\{tabular\}}
\NormalTok{\textbackslash{}end\{table\}}
\end{Highlighting}
\end{Shaded}

\end{frame}

\begin{frame}[fragile]{So import stuff (stata do-file example) !}

\small

\begin{Shaded}
\begin{Highlighting}[]
\NormalTok{sysuse auto                             }\CommentTok{# load car data set}
\NormalTok{regress mpg foreign weight               }\CommentTok{# first regression}
\NormalTok{eststo linear                      }\CommentTok{# store first regression}
\NormalTok{gen weight_sqr =}\StringTok{ }\NormalTok{weight*weight             }\CommentTok{# Quadratic term}
\NormalTok{regress mpg foreign weight weight_sqr      }\CommentTok{# 2nd regression}
\NormalTok{eststo quadratic                  }\CommentTok{# store second regression}
\NormalTok{esttab linear quadratic /}\ErrorTok{//}\StringTok{          }\CommentTok{# write to output file}
\StringTok{    }\NormalTok{using }\StringTok{"$\{outputfiles\}Results.tex"}\NormalTok{, /}\ErrorTok{//}
\StringTok{    }\KeywordTok{star}\NormalTok{(+}\StringTok{ }\FloatTok{0.1} \NormalTok{*}\StringTok{ }\FloatTok{0.05} \NormalTok{**}\StringTok{ }\FloatTok{0.01}\NormalTok{) replace }\KeywordTok{b}\NormalTok{(%}\FloatTok{9.}\NormalTok{4f) /}\ErrorTok{//}
\StringTok{    }\NormalTok{se r2 }\KeywordTok{scalars}\NormalTok{(}\StringTok{"F"}\NormalTok{) label }\KeywordTok{keep} \NormalTok{($covariates) /}\ErrorTok{//}
\StringTok{    }\KeywordTok{title}\NormalTok{(}\StringTok{"Dep = Milles per Gallon"}\NormalTok{) /}\ErrorTok{//}
\StringTok{    }\NormalTok{booktabs }\KeywordTok{alignment}\NormalTok{(D\{.\}\{.\}\{-}\DecValTok{1}\NormalTok{\}) nogaps}
\end{Highlighting}
\end{Shaded}

\end{frame}

\section{Making slides}\label{making-slides}

\begin{frame}{Pros and cons}

\begin{itemize}
\item
  Cons:

  \begin{itemize}
  \itemsep1pt\parskip0pt\parsep0pt
  \item
    Not as quick out of the box as PowerPoint (powerphluff)
  \item
    Typically beamer package which makes all things look alike

    \begin{itemize}
    \itemsep1pt\parskip0pt\parsep0pt
    \item
      Enforces some things (e.g., limited space for tables)
    \end{itemize}
  \end{itemize}
\item
  Pros:

  \begin{itemize}
  \itemsep1pt\parskip0pt\parsep0pt
  \item
    Once created, similar on all versions/operating machines
  \item
    You need to spend more time thinking
  \item
    Better \texttt{.pdf} handling
  \item
    Reuse of equations or code in general
  \item
    There is a kind of a philosophy behind it
  \end{itemize}
\end{itemize}

\begin{quote}
\href{http://users.ha.uth.gr/tgd/pt0501/09/Tufte.pdf}{The Cognitive
Style of PowerPoint} (Edward Tufte)
\end{quote}

\end{frame}

\begin{frame}[fragile]{Using beamer package}

\begin{Shaded}
\begin{Highlighting}[]
\NormalTok{\textbackslash{}documentclass\{beamer\}           }\CommentTok{% new document class}
    \NormalTok{\textbackslash{}usetheme\{Darmstadt\}                }\CommentTok{% new lay-out}
    \NormalTok{\textbackslash{}usecolortheme\{beaver\}         }\CommentTok{% new color scheme}
    \NormalTok{\textbackslash{}begin\{document\}           }\CommentTok{% begin document again}
       \CommentTok{% usually frames start with begin/end , except}
    \NormalTok{\textbackslash{}frame\{\textbackslash{}titlepage\} }
          \CommentTok{% Use section and subsection for slide menu}
    \NormalTok{\textbackslash{}section\{Where are we\}\textbackslash{}label\{where-are-we\} }
                              \CommentTok{% Frame and frame title}
    \NormalTok{\textbackslash{}begin\{frame\}\{Previous tutorial\}}
        \NormalTok{Still somewhat more theoretical ...}
        \NormalTok{$a^2 + b^2 = c^2$        }\CommentTok{% formula if you want}
    \NormalTok{\textbackslash{}end\{frame\}}
    \NormalTok{\textbackslash{}end\{document\}             }\CommentTok{% always end a document}
\end{Highlighting}
\end{Shaded}

\end{frame}

\begin{frame}{In conclusion}

\begin{itemize}
\item
  This tutorial is more a showcase
\item
  Pick out the stuff you appreciate most

  \begin{itemize}
  \itemsep1pt\parskip0pt\parsep0pt
  \item
    there is solution for almost everything
  \item
    but it requires time investment
  \item
    which only later will pay-off
  \end{itemize}
\item
  As things now develop there will be

  \begin{itemize}
  \itemsep1pt\parskip0pt\parsep0pt
  \item
    more ephasis on internet/blogging publishing (slightly more advance
    than Facebook but on the same par as Wordpress)

    \begin{itemize}
    \itemsep1pt\parskip0pt\parsep0pt
    \item
      including data and figures (dynamic infographics)
    \item
      minor role for \LaTeX~
    \end{itemize}
  \item
    For dead trees: \LaTeX~is still the best when editing/writing
    complex documents
  \end{itemize}
\end{itemize}

\end{frame}

\end{document}
