\documentclass[ignorenonframetext]{beamer}
\usetheme{Hannover}
\usecolortheme{dove}
\usefonttheme{structurebold}
\usepackage{amssymb,amsmath}
\definecolor{links}{HTML}{2A1B81}
\hypersetup{colorlinks,linkcolor=,urlcolor=links}
\usepackage[T1]{fontenc}
\usepackage[utf8]{inputenc}
\usepackage{microtype}
\usepackage{longtable, booktabs}
\usepackage[english]{babel}
\usepackage{pgf, microtype, booktabs, times, etex, dcolumn, minted}
\usepackage{tcolorbox}
\tcbuselibrary{minted,skins}

\newtcblisting{latexcode}{
	listing engine=minted,
	colback=bashcodebg,
	colframe=black!70,
	listing only,
	minted style=colorful,
	minted language=latex,
	minted options={linenos=true,texcl=true},
	left=1mm,
}
\definecolor{bashcodebg}{rgb}{0.85,0.85,0.85}
\beamertemplatenavigationsymbolsempty

\title{\LaTeX{} for Economics and Business Administration (Part II)}
\author{Thomas de Graaff}
\date{January 26, 2017}

\begin{document}
\frame{\titlepage}

\section{Introduction}\label{introduction}

\subsection{Recap}\label{introduction-1}

\begin{frame}{Previously}
	\begin{itemize}
		\item Pros and cons of \LaTeX{}
		\newline
		\item Why bother with learning \LaTeX{}
		\begin{itemize}
			\item for consistent/structured lay-out
			\item better automation of workflow
			\newline
		\end{itemize}
		\item Compiling, referencing, formula's, text control
	\end{itemize}
\end{frame}

\subsection{Agenda}

\begin{frame}{This session we look at}
\begin{itemize}
	\item Packages (controlling the preamble)
	\newline
	\item Figures (how to insert them?)
	\newline
	\item Tables (inserting plain tables)
	\newline
	\item Automatizing tables (complex tables)
	\newline
	\item Better looking references
	\newline
	\item Making slides
	\newline
\end{itemize}
	Note: we will only touch upon these subjects
\end{frame}

\section{Packages, packages, and packages}
\begin{frame}[fragile]{The use of packages}
	
	\begin{itemize}
		\item	Typically, packages are used to change appearance
		\item	To ensure no errors, usually opt for the full installation or have access to internet
		\item There are lots of them, see \href{http://www.ctan.org}{CTAN}
		\item	Often used packages
			\begin{itemize}
			\item amsmath, graphicx, subfig, marvosym, microtype, booktabs, lipsum,	pdflscape, fullpage, natbib
		\end{itemize}
		\item	format:
\scriptsize
\begin{latexcode}
\usepackage[colorlinks=true,citecolor=magenta,
            urlcolor=magenta]{hyperref}
\end{latexcode}
	\end{itemize}
\end{frame}

\begin{frame}[fragile]{The use of classes}
	\begin{itemize}
		\item	Typically one uses the \texttt{article} class
		\item	However, there is as well a \texttt{book}, \texttt{mininal},
		\texttt{report}, \texttt{beamer} class etcetera
		\item	Specific user written classes are \texttt{memoir} and	\texttt{elsarticle} classes
		\item	Classes come with options such as
\begin{latexcode}
\documentclass[12pt, a4paper]{article}
\end{latexcode}
	\end{itemize}
\end{frame}

\begin{frame}[fragile]{General structure}
	\begin{latexcode}
\documentclass[twocolumn, a4paper]{article}
% Preamble: how should it look like
\usepackage{multicol, lipsum}
\usepackage[english, german]{babel}
\begin{document}
	% Body: the real contents
	\lipsum
\end{document}
	\end{latexcode}
\end{frame}

\subsection{references}

\begin{frame}[fragile]{natbib \& biblatex}
	Default format is BibTeX---customizable (however limited). Default is good (except: use natbib!)
	\newline
	If you want to customize quite a lot: biblatex package!
	
\begin{latexcode}
\usepackage[style= authoryear-icomp,  
            backend=bibtex,
            natbib=true,
            firstinits=true,
            backref=true,
            maxnames=2,
            maxbibnames=10]
            	{biblatex}
\bibliography{mybib.bib}

\printbibliography
\end{latexcode}
\end{frame}

\section{Figures}

\begin{frame}[fragile]{Figures}
	Figures/graphs and tables in a floating environment\\
	\begin{latexcode}
\begin{figure}[htbp!]}
	\center
	\includegraphics{ligatures_latex}
	\caption{A figures about ligatures}
	\label{fig:ligatures}
\end{figure}
\end{latexcode}
Figures can be \texttt{.pdf}, \texttt{.jpg}, \texttt{.png} and a whole lot of other types (but not bitmaps!)
\end{frame}

\section{Tables}

\begin{frame}[fragile]{Tables}
\begin{latexcode}
\begin{table}[t!]
	\caption{This is the caption}
	\begin{tabular}{lcr}
		\hline
		first & row & data \\
		second & row & data \\
		\hline
	\end{tabular}
	\label{tab:example}
\end{table}
\end{latexcode}
\end{frame}

\subsection{Referencing}

\begin{frame}[fragile]{Referencing}
	Internal references are a breeze
	\begin{latexcode}
\label{}	% Label something
\ref{}	% Refer to that
\footnote{}	% Add footnote
\thanks{}	% For in title
	\end{latexcode}
\end{frame}

\begin{frame}[fragile]{dcolumn and booktabs package}
\small
\begin{latexcode}
\usepackage{booktabs, dcolumn} % in preamble
\newcolumntype{d}{D{.}{.}{2}}  % in preamble
\begin{table}[t!]
	\caption{This is the caption}
	\begin{tabular}{ldd}
		\toprule
		Student & Grade 1 & Grade 2 \\
		\midrule
		Mike    & 7.8 & 9   \\
		Andrea  & 6   & 8.2 \\
		\bottomrule
	\end{tabular}
	\label{tab:example2}
\end{table}
\end{latexcode}	
\end{frame}

\subsection{Automizing tables}

\begin{frame}[fragile]{Some R code}
\tiny
		\begin{minted}{r}
library(texreg)

control <- c(4.17, 5.58, 5.18, 6.11, 4.50, 4.61, 5.17, 4.53, 5.33, 5.14)
treated <- c(4.81, 4.17, 4.41, 3.59, 5.87, 3.83, 6.03, 4.89, 4.32, 4.69)
group <- gl(2, 10, 20, labels = c("Control", "Treated"))
weight <- c(control, treated)
m1 <- lm(weight ~ group - 1)
m2 <- lm(weight ~ group)

texreg(list(m1, m2), dcolumn = TRUE, booktabs = TRUE, file = "Table.tex",
use.packages = FALSE, label = "tab:3", caption = "Two linear models.",
float.pos = "hb!")
		\end{minted}
\end{frame}

which saves a file "Table.tex" to the same directory

\begin{frame}[fragile]{Statistical output}
Now 
\begin{latexcode}

% Table created by stargazer v.5.2.2 by Marek Hlavac, Harvard University. E-mail: hlavac at fas.harvard.edu
% Date and time: Sat, Jan 25, 2020 - 13:34:28
% Requires LaTeX packages: dcolumn 
\begin{table}[!htbp] \centering 
  \caption{My experiment} 
  \label{} 
\begin{tabular}{@{\extracolsep{5pt}}lD{.}{.}{-3} D{.}{.}{-3} } 
\\[-1.8ex]\hline 
\hline \\[-1.8ex] 
 & \multicolumn{2}{c}{\textit{Dependent variable:}} \\ 
\cline{2-3} 
\\[-1.8ex] & \multicolumn{2}{c}{weight} \\ 
\\[-1.8ex] & \multicolumn{1}{c}{(1)} & \multicolumn{1}{c}{(2)}\\ 
\hline \\[-1.8ex] 
 groupControl & 5.032^{***} &  \\ 
  & (0.209) &  \\ 
  groupTreated & 5.761^{***} & 0.729^{**} \\ 
  & (0.209) & (0.295) \\ 
  Constant &  & 5.032^{***} \\ 
  &  & (0.209) \\ 
 \hline \\[-1.8ex] 
Observations & \multicolumn{1}{c}{20} & \multicolumn{1}{c}{20} \\ 
R$^{2}$ & \multicolumn{1}{c}{0.987} & \multicolumn{1}{c}{0.253} \\ 
\hline 
\hline \\[-1.8ex] 
\textit{Note:}  & \multicolumn{2}{r}{$^{*}$p$<$0.1; $^{**}$p$<$0.05; $^{***}$p$<$0.01} \\ 
\end{tabular} 
\end{table} 

\end{latexcode}
produces:

\begin{scriptsize}
	
% Table created by stargazer v.5.2.2 by Marek Hlavac, Harvard University. E-mail: hlavac at fas.harvard.edu
% Date and time: Sat, Jan 25, 2020 - 13:34:28
% Requires LaTeX packages: dcolumn 
\begin{table}[!htbp] \centering 
  \caption{My experiment} 
  \label{} 
\begin{tabular}{@{\extracolsep{5pt}}lD{.}{.}{-3} D{.}{.}{-3} } 
\\[-1.8ex]\hline 
\hline \\[-1.8ex] 
 & \multicolumn{2}{c}{\textit{Dependent variable:}} \\ 
\cline{2-3} 
\\[-1.8ex] & \multicolumn{2}{c}{weight} \\ 
\\[-1.8ex] & \multicolumn{1}{c}{(1)} & \multicolumn{1}{c}{(2)}\\ 
\hline \\[-1.8ex] 
 groupControl & 5.032^{***} &  \\ 
  & (0.209) &  \\ 
  groupTreated & 5.761^{***} & 0.729^{**} \\ 
  & (0.209) & (0.295) \\ 
  Constant &  & 5.032^{***} \\ 
  &  & (0.209) \\ 
 \hline \\[-1.8ex] 
Observations & \multicolumn{1}{c}{20} & \multicolumn{1}{c}{20} \\ 
R$^{2}$ & \multicolumn{1}{c}{0.987} & \multicolumn{1}{c}{0.253} \\ 
\hline 
\hline \\[-1.8ex] 
\textit{Note:}  & \multicolumn{2}{r}{$^{*}$p$<$0.1; $^{**}$p$<$0.05; $^{***}$p$<$0.01} \\ 
\end{tabular} 
\end{table} 

\end{scriptsize}
\end{frame}

\section{Slides}

\subsection{Beamer class}

\begin{frame}[fragile]{Slides}
	Slides are typically made with the beamer class
	\begin{latexcode}
\documentclass{beamer}
\title{Another lecture}
\author{By a wisecrack lecturer}

\begin{document}
  \frame{\titlepage}
	
  \begin{frame}{Introduction}
%	Typically a quote from a long 
%	dead philosopher that should
%	make the lecturer look smart
%	but usually does not.
  \end{frame}		
\end{document}
\end{latexcode}	
\end{frame}

\begin{frame}[fragile]{Beamer style}
	You can change the beamer style by:
	\begin{latexcode}
\usetheme{Hannover}
\usecolortheme{dove}

% to remove those navigation symbols
\beamertemplatenavigationsymbolsempty
	\end{latexcode}
\href{https://www.hartwork.org/beamer-theme-matrix/}{(https://www.hartwork.org/beamer-theme-matrix/} gives all possible combinations
\end{frame}

\section{Conclusion}

\begin{frame}{In conclusion}
	\begin{itemize}
		\item \LaTeX{} is a very powerful structured language especially suitable for
		\begin{itemize}
			\item large complex documents;
			\item documents with many formula's.
			\newline
		\end{itemize}
	\item Big advantage: you really need to think
	\newline
	\item Not for every one; steep learning curve, but
	\newline
	\item large community (google it)
	\newline
	\item Markup language (especially, Markdown) becomes more and more wide-spread: \LaTeX{} is a good start
	\end{itemize}
\end{frame}

\end{document}